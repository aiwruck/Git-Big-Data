\documentclass{article}
\usepackage[hyphens]{url}
\usepackage{amsmath, amssymb}
\usepackage{geometry}
\usepackage{graphicx}
\usepackage{xcolor}
\usepackage{hyperref}

\title{Problem Set 0}
\author{Alice Wruck}
\date{2023-09-11}

\begin{document}
\maketitle

\section{Section 1}
I wanted to take this class as I think it would be useful for my job post-grad. This past summer I was a Summer Analyst at Cornerstone Research which is an Economic Consulting firm in Boston, MA which is the same firm and job I will be returning to after graduating. A big piece of this position was using R. Given that prior to this summer I had never used R, it was a steep learning curve. I was the "checker" side of the code for both of my cases. This entailed having large (often messy) data sets and being told what I wanted the code to do, but not how to get there. The idea being if the "doer" gave me too much information on how to do it, I could do it the same way they did, but the same incorrect way. Seeing as I had no R experience before this, I used google a lot. I definitely learned a lot and feel decently confident in my R skills but think this class will take me to another level of confidence and ability. With that being said, my goal for this course is to become even more comfortable using R as I know I will be using it a lot after school as well. For my project for this course, I think I would be interested in the second-hand market for luxury goods. (This is the topic of my thesis, so when the time comes to get started on this project, I will reach out to both my thesis advisor and you to see if that is okay with both. If not, I will work on a different topic.) More specifically, the rise of participation in the market the mechanisms through which the goods are sold and priced. Along a similar line, I would be interested in stock performance of public companies that sell luxury goods. Additionally, the way their performance changes (if it changes) and whether or not these mergers create value in the industry. 


\section{Equation}

\[a^2 + b^2 = c^2 \]

\end{document}